% \chapter*{Conclusion et Bilan personnel}

% Insérer ici le texte de votre conclusion
% La conclusion doit porter sur les travaux que vous avez réalisés. Quelles sont  les questions que vous vous êtes posées et les réponses que vous y avez apportées ? Pensez à porter un œil critique et à élargir votre réflexion.
% Il est important de souligner vos acquis personnels et professionnels et de dresser un bilan de cette expérience en le mettant en perspective de votre formation et de votre projet professionnel.
% N’oubliez pas d’effacer ce texte quand vous n’en aurez plus besoin.

\chapter*{Conclusion}

In conclusion, we exploited a new experimental paradigm using minimalist sequences of empty intervals with MEG recordings to investigate the neuroanatomy and neurodynamics of working memory durations. Our analyses highlight the importance of the alpha band, somatosensory areas and occipital cortex in working memory. To obtain these results, we integrated the algorithms for time-frequency analysis and visualization of the contrasts in the source space into an open-source pipeline. In order to make the pipeline usable and practical, considerable work has been done in optimizing the running time of the pipeline, combining advanced software engineering techniques with techniques from statistical learning theory. The implemented algorithms allow to combine the performance needed to analyze data of such dimensions with the interpretability required for the analysis of brain signals.
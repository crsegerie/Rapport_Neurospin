\chapter*{The neuroanatomical signature of time in Working Memory}

% Il m'a fallu un temps conséquent pour me former au début du stage à l'analyse en EEG. 
% Au MVA, j'avais déjà fait des neurosciences, ainsi que la théorie physique des EEG et MEG à l'aide du cours d' \href{http://math.ens-paris-saclay.fr/version-francaise/formations/master-mva/contenus-/imagerie-fonctionnelle-cerebrale-et-interface-cerveau-machine-161979.kjsp?RH=1242430202531}{Imagerie fonctionnelle cérébrale et interface cerveau machine}, au cours duquel j'ai déjà manipulé les interfaces cerveau machines par exemple en
% \href{https://github.com/crsegerie/bci_competition}{réimplémentant} la solution des vainqueurs de la compétition \href{https://www.kaggle.com/c/inria-bci-challenge}{inria-bci-challenge}. Mais en réalité, lors de cette compétition kaggle, les participants ont été épargnés du préprocessing du traitement de la donnée des EEG : les organisateurs du kaggle avaient déjà nettoyé la donnée qui était alors utilisable par n'importe quel praticien en machine learning. Mais il existe tout un savoir faire afin de nettoyer la donnée, et ce nettoyage et cette analyse est presque plus technique que du machine learning conventionnel.



% \begin{comment}
% \section{Théorie}
% voir le cours de gramfort. Voir le cours du mva.
% \section{L'écosysteme MNE - Bids}
% \subsection{mne-python}
% \subsection{mne-bids-pipeline}
% \subsection{mne-bids}
% \end{comment}

\section{Prélude}

J'ai séparé ce rapport en deux parties qui suivent grossièremetnt l'ordre chronologiqe de mon stage.

Le premier chapitre donne le contexte scientifique de mon stage. Cete partie ne consitutue pas compte rendu de mon travail mais plutot de l'environnement dans lequel je m'inserre et des connaissances prerequises pour comprendre par la suite mon travail. L'activité scientifique n'est plus une activité solitaire, surtout en neuroscience, ou les outils utilisés reposent dsur des années de travail collectifs. Cette partie rend donc hommage au épaules de géants ssur lesquelles je me suis basé pendant mon stage. J'explique notemment dans ce chapitre les tenant et aboutissants du papier sémin al/ Pilot study sur lequel je me repose. Mais ce papier initial ne permettait pas de voir en détail la structure du cerveau pendant l'accomplissaement de la tache. Ce chapitre pourrait consitituer l'introduction d'un papier publié à la suite de ce stage.

Le second chapitre est une exposition de mon travail. J'y expose en détail la reproduction du papier séminal : "Abstracting time in Working Memory" de Sophie Herbst en utilisant cette fois ci la MEG. Nous avons reproduit les résultat de la Pilot study mais en utilisant l'imagerie par MEG afin d'analyser les parties du cerveau à l'oeuvre lors de l'utilisation de la mémoire de trvail. La MNE-bis -pipeline, qui est une pipeline d'analyse automatique de donnée MEEG consititua le socle de notre analyse. Mais afin de pouvoir réellement analyser la donnée cérébrale, il a fallut adapter la mne-bids-pipeline en utilisant de nouveaux algorithmes afin de répondre aux exigence de notre paradigme experimental. Cette partie pourrait constituer la partie methode, resultats et discussion d'un futur papier.


\chapter{Scientific Context}
Ici mettre tout ce que l'on pouvait deviner avant de faire le stage

\section{Lire la feuille de stage}

Why this stage
Mes connaissances préalables: MVA cours imagerie fonctionnelel cerebrale + Neuroscience + imagerie médiacle
L'environnement : les différentes équipes

% Les equipes
% - Inria Parietal 
% - Neurospin, Cognition and Brain dynamics.

Les librairies
- L'écosysteme MNE
    - mne-python, bids, mne-bids-pipeline

Theorie
- Les modèles de clocks cerebrales
- EEG
- Paradigme de la Pilot study
- Pourquoi EEG plutot que FMRI
- Donner le but du Stage : 
    Reproduire cette study mais avec la MEG
- Glossaire

\chapter{Ma contribution: replication mais avec MEG}

% Setup
% - renvoyer au document de l'ensemble des choses que j'auraient aimés savoir dés le début

% Deux études : 
% - Welcome study
% - Sample

\section{Preprocessing}

- Bids Pipeline
- Premiereres PR necessité pourquoi ?
    premiere contri open source
    avec une utilité
- Dataset description
- Tableau des proprocessing
- Cheat Cheet
- Raconter la pipeline BIDS, utilité
- Limites:
    - Nombre de participants

\section{Sensor Space}

- Pipeline classique raconter les steps
- Motivation : Insuffisance de la pipeline automatique: beoin d'une nouvelle step
- Theorie algo:
    - CSP
    - Permutation tests
    - subtilités mathématiques
    - Ingénieurie logicielle et recherche ?
- Anexes : Implémentation et choix d'architecte logiciel
    - optimisation du temps
        reduction dimensionalité
        reduction du nombre de channels
    - CV, decimation, niquist
    - Interaface avec le reste de la pipeline, openscience, reproductibilité
- Results:
    - Images
- Discussion
    - resultats significatifs alors que dans la tête
- Limites + Discu

\section{Source Space}

- Generalités source space
Nouveau script: Contrast in source space
    - Motivation : Insuffiscance de pipeline classique
    - faire le Schéma
- Results:
    - Interpretation sur Freeview
- Discussion
- Annexes : Subtilité : 
    Neuroscience : quest ce qui constitue du bruit ? Quel choix de la matrice de covariance
    Subtilités mathématiques :  commutativité Log + moyenne
Limites:
    - 





une fois que l'on fini le plan on appelle MM et on lui explique, et on regarde par quoi on commence l'explication +  on enregistre dictaphone
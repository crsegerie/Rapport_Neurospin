\chapter*{Introduction}

Texte de l'introduction
L’introduction ne se rédige pas en début de travail de rédaction mais à la fin, ou au moins lorsque l’on a un plan détaillé en tête.
L’introduction est, en quelque sorte, la consigne de lecture. Elle doit à la fois définir le sujet ou l’orientation de votre rapport et annoncer le plan, c’est-à-dire la logique de la construction.


\chapter*{Host organizations}

My internship was a bit particular in the sense that it took place between two research institutes. I worked in parallel with :
\begin{itemize}
    \item Inria, within the Parietal team
    \item and Neurospin, in the Cognition and Brain Dynamics team.
\end{itemize} 

\section*{Inria}

The French National Institute for Research in Computer Science and Control (Inria) is a French national research institution focused on computer science and applied mathematics. Inria employs 3800 people, including 2800 researchers and doctoral students located all over France.

\section*{Inria - Parietal}

The parietal team is a laboratory of about 30 people dedicated to Learning brain structure, function and variability from neuroimaging data. The parietal team focuses on mathematical methods for statistical modeling of brain functions from neuroimaging data (fMRI, MEG, EEG), with a particular interest in machine learning techniques, applications to human cognitive neuroscience and scientific software development.

Within inria parietal, I have particularly worked with the team of developers now \textbf{mne-python}, and \textbf{mne-bids-pipeline}, especially with Richard Höchenberger and Alexandre Gramfort. But I was able to be present at the team presentations every Tuesday afternoon during the parietal talks. Every Tuesday afternoon, we had team presentations, which gave us the opportunity to follow the work of the other members of the parietal team, and to present our work.

\section*{Neurospin}

Neurospin is a brain imaging center in Saclay, south of Paris. The INSERM-CEA Cognitive Neuroimaging Unit, comprises five teams:

\begin{itemize}
    \item Languages of the Brain, which attempts to answer the question: Why are we the only species with a sophisticated communication system?
    \item Neuroimaging of Development, which studies human cognitive development in infants and children, both structurally and functionally, and aims to develop new imaging techniques appropriate for human infants.
    \item Neuromodulation which focuses on brain function in primates.
    \item Computational Brain, which studies the different functions of the human brain from the perspective of computation and information coding.
    \item \textbf{Cognition and Brain Dynamics} which is interested in the processing of multisensory information, their temporal organization and in particular the representation of temporal information in the human brain, using magnetoencephalography (MEG) as the main methods.
\end{itemize}

I did my internship in this last team: Cognition and brain dynamics with Sophie Herbst and Virginie van Wassenhove but I was also able to interact with some of the people from the other teams, especially during the Friday afternoon meetings that allows the 5 teams to get together.


% \chapter*{Présentation des encadrants}


% J'ai été encadré par Alexandre Gramfort, Sophie Herbst, ainsi que Richard Höchenberger.

\chapter*{Introduction}

Texte de l'introduction
L’introduction ne se rédige pas en début de travail de rédaction mais à la fin, ou au moins lorsque l’on a un plan détaillé en tête.
L’introduction est, en quelque sorte, la consigne de lecture. Elle doit à la fois définir le sujet ou l’orientation de votre rapport et annoncer le plan, c’est-à-dire la logique de la construction.


\chapter*{Présentation de l’organisme d’accueil}

Mon stage était un peu particulier dans le sens ou celui-ci s’est déroulé entre deux instituts de recherches. J’ai travaillé en parallèle avec :
\begin{itemize}
    \item l’INRIA, au sein de l’équipe Parietal
    \item et Neurospin, au sein de l’équipe Cognition and Brain Dynamics.
\end{itemize} 

\section{Inria}

L'Institut national de recherche en informatique et en automatique (Inria) est un établissement national de recherche français axé sur l'informatique et les mathématiques appliquées. Il a été créé sous le nom d'Institut de recherche en informatique et en automatique (IRIA), qui faisait partie du Plan Calcul. Son premier site était les locaux historiques du SHAPE (commandement central des forces militaires de l'OTAN), qui est toujours utilisé comme siège principal d'Inria. En 1980, l'IRIA est devenu l'INRIA.
Inria est un Établissement public de recherche scientifique et technique (EPST) placé sous la double tutelle du ministère de l'Éducation nationale, de l'Enseignement supérieur et de la Recherche et du ministère de l'Économie, des Finances et de l'Industrie.
 
Inria dispose de huit centres de recherche répartis en France (à Bordeaux, Grenoble-Inovallée, Lille, Nancy, Paris-Rocquencourt, Rennes, Saclay, et Sophia Antipolis). Inria emploie 3800 personnes. Parmi elles, on compte 1300 chercheurs, 1000 doctorants et 500 postdoctorants.

\section{Inria - Parietal}

Learning brain structure, function and variability from neuroimaging data.

The Parietal team focuses on mathematical methods for statistical modeling of brain function using neuroimaging data (fMRI, MEG, EEG), with a particular interest in machine learning techniques, applications to human cognitive neuroscience, and scientific software development.


Au sein de l'inria parietal, j'ai particulièrement travaillé avec l'équipe de deellopeurs maintenant mne-python, et mne-bids-pipeline.

\section{Neurospin}

Neurospin est un centre d'imagerie cérébrale à Saclay au Sud de Paris. The INSERM-CEA Cognitive Neuroimaging Unit, directed by Stanislas Dehaene, Professor at Collège de France, comprises five teams:

\begin{itemize}
    \item Languages of the Brain : Directed by Christophe Pallier, whi try to answer the question : Why are we the only species with a sophisticated communication system using a combinatorial language, as well as a capacity to develop languages in many other domains, such as music or mathematics?
    \item Neuroimaging of Development. Directed by Ghislaine Dehaene-Lambertz, DRCE CNRS
    "We examine human cognitive development in infants and children, both at the structural and functional levels, and to develop new imaging techniques adapted to human infants.
    You can follow our work here (in French Mon cerveau à l’école)".
    \item Neuromodulation : Directed by Béchir Jarraya, Professor at University of Versailles Paris Saclay: "We analyze the primate brain functions, and evaluate their modulation by pharmacological agents or electrical neurostimulations.".
    \item The Computational Brain : Directed by Florent Meyniel, CEA
    Our team studies various functions of the human brain from the viewpoint of computations. Our goal is to propose quantitative models of brain functions, algorithms for the processing of information and identity the neural codes that subtend them.
    \item \textbf{Cognition and Brain Dynamics}: Directed by Virginie van Wassenhove, DR CEA
    We evaluate the processing of multisensory information, its temporal organization and particularly the representation of temporal information in the human brain, using magnetoencephalography (MEG) as a principal methods.
\end{itemize}

J'ai fait mon stage au sein de cette dernière équipe : Cognition and brain dynamics.


\chapter*{Présentation des encadrants}


J'ai été encadré par Alexandre Gramfort, Sophie Herbst, ainsi que Richard Höchenberger.

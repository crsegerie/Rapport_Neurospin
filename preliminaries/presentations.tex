
\chapter*{Host organizations}

My internship was a bit particular because it took place between two research institutes. I worked in parallel with :
\begin{itemize}
    \item Inria, within the Parietal team
    \item and Neurospin, in the Cognition and Brain Dynamics team.
\end{itemize}

\section*{Inria}
The French National Institute for Research in Computer Science and Control (Inria) is a French national research institution focused on computer science and applied mathematics. Inria employs 3800 people, including 2800 researchers and doctoral students located all over France.

\section*{Inria - Parietal}

The parietal team is a laboratory of about 30 researchers dedicated to learning brain structure, function and variability from neuroimaging data. The parietal team focuses on mathematical methods for statistical modeling of brain functions from neuroimaging data (fMRI, MEG, EEG), with a particular interest in machine learning techniques, applications to human cognitive neuroscience and scientific software development.

Within Inria parietal, I have particularly worked with the team of developers now \textbf{MNE-Python}, and \textbf{MNE-BIDS-Pipeline}, especially with Richard Höchenberger and Alexandre Gramfort. But I was able to be present at the team presentations every Tuesday afternoon during the parietal talks. Every Tuesday afternoon, we had team presentations, which gave us the opportunity to follow the work of the other members of the parietal team, and to present our work.

\section*{Neurospin}

Neurospin is a brain imaging center in Saclay, south of Paris. The INSERM-CEA Cognitive Neuroimaging Unit, comprises five teams:

\begin{itemize}
    \item Languages of the Brain, which attempts to answer the question: Why are we the only species with a sophisticated communication system?
    \item Neuroimaging of Development, which studies human cognitive development in infants and children, both structurally and functionally, and aims to develop new imaging techniques appropriate for human infants.
    \item Neuromodulation which focuses on brain function in primates.
    \item Computational Brain, which studies the different functions of the human brain from the perspective of computation and information coding.
    \item \textbf{Cognition and Brain Dynamics} which focuses on the processing of multisensory information, their temporal organization and in particular the representation of temporal information in the human brain, using magnetoencephalography (MEG) as the main methods.
\end{itemize}

I did my internship in this last team: Cognition and brain dynamics with Sophie Herbst, but I was also able to interact with some people from the other teams, especially during the Friday afternoon meetings that allow the 5 teams to get together.


% \chapter*{Présentation des encadrants}

% J'ai été encadré par Alexandre Gramfort, Sophie Herbst, ainsi que Richard Höchenberger.


\chapter*{Introduction}

% Texte de l'introduction
% L’introduction ne se rédige pas en début de travail de rédaction mais à la fin, ou au moins lorsque l’on a un plan détaillé en tête.
% L’introduction est, en quelque sorte, la consigne de lecture. Elle doit à la fois définir le sujet ou l’orientation de votre rapport et annoncer le plan, c’est-à-dire la logique de la construction.

% Il m'a fallu un temps conséquent pour me former au début du stage à l'analyse en EEG. 
% Au MVA, j'avais déjà fait des neurosciences, ainsi que la théorie physique des EEG et MEG à l'aide du cours d' \href{http://math.ens-paris-saclay.fr/version-francaise/formations/master-mva/contenus-/imagerie-fonctionnelle-cerebrale-et-interface-cerveau-machine-161979.kjsp?RH=1242430202531}{Imagerie fonctionnelle cérébrale et interface cerveau machine}, au cours duquel j'ai déjà manipulé les interfaces cerveau machines par exemple en
% \href{https://github.com/crsegerie/bci_competition}{réimplémentant} la solution des vainqueurs de la compétition \href{https://www.kaggle.com/c/inria-bci-challenge}{inria-bci-challenge}. Mais en réalité, lors de cette compétition kaggle, les participants ont été épargnés du préprocessing du traitement de la donnée des EEG : les organisateurs du kaggle avaient déjà nettoyé la donnée qui était alors utilisable par n'importe quel praticien en machine learning. Mais il existe tout un savoir faire afin de nettoyer la donnée, et ce nettoyage et cette analyse est presque plus technique que du machine learning conventionnel.

% \begin{comment}
% \section{Théorie}
% voir le cours de gramfort. Voir le cours du mva.
% \section{L'écosysteme MNE - Bids}
% \subsection{MNE-Python}
% \subsection{MNE-BIDS-Pipeline}
% \subsection{MNE-BIDS}
% \end{comment}

I divided this report into four parts.

The first chapter gives the scientific context of my internship. This part is an account of the environment in which I am inserted and of the knowledge required to understand my work hereafter. Science is no longer a solitary activity, especially in neuroscience, where the techniques used derive from years of collaborative work. I explain in this chapter the ins and outs of the pilot study paper \cite{herbst2021abstracting} on which I rely. The goal of my internship is to complete this pilot study by analyzing magnetoencephalography (MEG) data, by finding appropriate mathematical methods to analyze the MEG data, and to implement these mathematical methods in an open source automatic analysis pipeline, the MNE-BIDS-Pipeline. This pipeline aims to allow future cognitive science researchers to analyze electroencephalogram data in a snap, while promoting replicability and open science. This part could constitute the introduction part of a future paper.


The second chapter presents the more technical aspects of my contribution to the reproduction of the pilot study. I detail the method used to analyze the parts of the brain in action during the use of the working memory. We used the MNE-BIDS-Pipeline to analyze the data. But we had to adapt the MNE-BIDS-Pipeline by using new algorithms to meet the requirements of our experimental paradigm. An important part of my work has been allocated to implement in open-source these methods. This part could constitute the method part of a future paper.


The third part presents the obtained results. The beginning of a discussion is also outlined.

An appendix provides more information on the algorithms used and on some mathematical subtleties. An important point is made on the optimization of the computation time. Indeed, even if the optimization of the computation time does not change the results, it was critical in the practical feasibility of our project.
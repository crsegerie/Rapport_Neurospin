\chapter*{\centering Fiche de synthèse}
\begin{itemize}
    \item Type de stage : Stage de recherche
    \item Année : 2021
    \item Auteur (Nom, prénom) : Charbel-Raphaël Segerie
    \item Formation 2ème année (IMI, GI, SEGF, etc.) : IMI
    \item Titre du rapport : The neuroanatomical signatures of time in working memory
    \item Titre en français : Les signatures neuroanatomiques du temps dans la mémoire de travail
    \item Organisme d’accueil : Inria Parietal / Neurospin
    \item Pays d’accueil : France
    \item Responsable de stage : Eric Duceau.
    \item Mots-clés caractérisant votre rapport (4 à 5 mots maximum) : MEG, time perception, time-frequency analysis, working memory.
\end{itemize}

% \chapter*{\centering Remerciements}

% S'il y a lieu, insérer ici votre texte de remerciements.


\chapter*{\centering Résumé}

L'évaluation de la durée est l'une des capacités fondamentales du cerveau. Cependant, nous ne savons pas comment cette information est stockée en mémoire. Ici, nous utilisons un paradigme de reproduction temporelle récemment développé pour étudier la dynamique neuronale du temps dans la mémoire de travail. Les participants ont écouté des séquences non rythmiques composées d'intervalles temporels, qu'ils devaient ensuite reproduire le plus fidèlement possible. Nous avons fait varier la longueur totale des séries temporelles ainsi que le nombre d'intervalles (n-item) les composant. L'expérience a été menée et enregistrée par magnétoencéphalographie ($N=24$). Au cours de mon stage, j'ai développé des méthodes mathématiques pour analyser ces enregistrements. Nous procédons en deux étapes : 1. Nous procédons à une analyse temps-fréquence des données et nous trouvons les principaux clusters informationnels. 2. Nous utilisons les clusters temps-fréquence identifiés pour analyser les données dans l'espace source. Nos résultats suggèrent que la mémoire de travail temporelle est stockée principalement dans la bande alpha (8-14 Hz), et qu'elle est localisée dans l'aire somatosensorielle complétée par le cortex occipital.

Mots clés :
Mémoire de travail, apprentissage statistique, magnétoencéphalographie, analyse temps-fréquence, signaux multivariés.

\chapter*{\centering Abstract}

Evaluating duration is one of the fundamental capacities of the brain. Yet, we do not know how this information is stored in memory. Here, we use a recently developed temporal reproduction paradigm to investigate the neural cynamic of time in working memory. Participants listened to non-rhythmic sequences composed of temporal intervals, which they then had to reproduce. We varied the total length of the temporal series as well as the number of intervals (n-item) composing them. The experiment was conducted and recorded by magnetoencephalography ($N=24$). During my internship, I developed mathematical methods to analyze these recordings. We proceed in two steps: 1. We proceed to a time-frequency analysis of the data, and we find the main informational clusters. 2. We use the identified time-frequency clusters to analyze the data in the source space. Our results suggest that temporal working memory is stored predominantly in the alpha band (8-14 Hz), and that it is located in the somatosensory area complemented by the occipital cortex.

Keywords:
Working memory, Statistical learning, Magnetoencephalography, Time-frequency analysis, Multivariate signals.

% \chapter*{\centering Synthèse du rapport en français}

% À utiliser dans le cas des rapports rédigés en anglais (4 pages minimum).
